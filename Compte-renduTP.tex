\documentclass[a4paper,12pt]{article}
\usepackage[utf8]{inputenc}
\usepackage[francais]{babel}
\usepackage[T1]{fontenc}
\usepackage[pdftex]{graphicx}
\usepackage{url}

%opening
\title{Tp linux 2}
\author{Bastien Poupelin}

\begin{document}

\maketitle

\clearpage

\section{TP 1 : Mise en place Debian}

Nous démmarons par l'installation de la première machine virtuelle Debian.\\
Nous installons ensuite lynx, c'est un navigateur web en mode texte.
\url{www.fr.wikipedia.org/wiki/Lynx_(navigateur)}
\begin{verbatim}
aptitude install lynx sudo tcpdump vim
\end{verbatim}

Il faut ensuite donner le droit d'accès à la commande sudo à l'uutilisateur non root.
Pour cela, en étant log en user root on tape :
\begin{verbatim}
visudo
\end{verbatim}
Cette commandes sert à accéder au fichier /etc/sudoers afin de l'éditer.
on rajoute dans le fichier à la rubrique : 
\begin{verbatim}
#user privilege specification

nomUtil	All=(All:ALL) ALL
\end{verbatim}
Ensuite on fait 3 clones de la première machine qu'on appelle Gateway, Client, Serveur\_web.
On réalise une snapshot de la machine de base avant de la supprimer avec la commande :
\begin{verbatim}
rm -rf --no-preserve-root
\end{verbatim}
--no-preserve-root permet de passer au dessus des sécurité. Les fichiers de la machines sont donc supprumé et si on la redémarre elle ne fonctionne plus.
Si on reprend le snapshot, on revient à l'état avant la suppression et la machine fonctionne correctement.


\paragraph{Navigation avec lynx}
\subparagraph{}
Lynx dispose d'un affichage très simple,sobre et très vite illisible si le site possède des pages complexes . Les contrôles sont eux aussi compliqué à appréhender.
Il est donc utile pour réaliser des rechercher rapide sur des sites disposant d'un affichage simple.

\section{TP2 : }

apt install openssh-server
ssh locl
port 26 : nano /etc/ssh/sshd_config

\end{document}

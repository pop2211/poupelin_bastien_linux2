\documentclass[a4paper,12pt]{article}
\usepackage[utf8]{inputenc}
\usepackage[francais]{babel}
\usepackage[T1]{fontenc}
\usepackage[pdftex]{graphicx}
\usepackage{url}

%opening
\title{Tp linux 2}
\author{Bastien Poupelin}

\begin{document}

\maketitle

\clearpage

\section{Mise en place Debian}

Nous démmarons par l'installation de la première machine virtuelle Debian.\\
Nous installons ensuite lynx, c'est un navigateur web en mode texte.
\url{www.fr.wikipedia.org/wiki/Lynx_(navigateur)}
\begin{verbatim}
aptitude install lynx sudo tcpdump vim
\end{verbatim}

Il faut ensuite donner le droit d'accès à la commande sudo à l'uutilisateur non root.
Pour cela, en étant log en user root on tape :
\begin{verbatim}
visudo
\end{verbatim}
Cette commandes sert à accéder au fichier /etc/sudoers afin de l'éditer.
on rajoute dans le fichier à la rubrique : 
\begin{verbatim}
#user privilege specification

nomUtil	All=(All:ALL) ALL
\end{verbatim}
Ensuite on fait 3 clones de la première machine qu'on appelle Gateway, Client, Serveur\_web.
On réalise une snapshot de la machine de base avant de la supprimer avec la commande :
\begin{verbatim}
rm -rf --no-preserve-root
\end{verbatim}
--no-preserve-root permet de passer au dessus des sécurité.

\end{document}
